\documentclass{beamer}

\usepackage{helvet}
\usepackage{hyperref, graphicx}
\usepackage{amsthm}
\usepackage{etoolbox}

\usetheme{default}
\setbeamertemplate{navigation symbols}{}
\AtBeginSection[ ]
{
\begin{frame}{Outline}
    \tableofcontents[currentsection]
\end{frame}
}

% Default fixed font does not support bold face
\DeclareFixedFont{\ttb}{T1}{txtt}{bx}{n}{11} % for bold
\DeclareFixedFont{\ttm}{T1}{txtt}{m}{n}{12}  % for normal - use in headings

% Custom colors
\usepackage{color}
\definecolor{TUGray}{RGB}{101,101,137}
\definecolor{TUBlack}{RGB}{30,0,0}
\definecolor{mygreen}{RGB}{45,111,63}
\definecolor{keywords}{RGB}{205,114,0}
\definecolor{comments}{RGB}{181,51,139}
\definecolor{strings}{RGB}{58,144,81}
\definecolor{numeric}{RGB}{66,110,176}
\definecolor{linos}{rgb}{0.4,0.4,0.4}
\definecolor{links}{rgb}{0,0.4,0.75}

\definecolor{bggray}{RGB}{232, 233, 235}

\usecolortheme[named=mygreen]{structure}
\setbeamercolor{normal text}{fg=TUBlack}\usebeamercolor*{normal text}

\setbeamercolor{codecol}{fg=TUGray!25!black,bg=bggray}

\hypersetup{colorlinks, linkcolor=links, urlcolor=links}

\usepackage[T1]{fontenc}
\usepackage[sfdefault,scaled=.85]{FiraSans}
\usepackage{newtxsf}

\usepackage{listings}

\newtoggle{InString}{}% Keep track of if we are within a string
\togglefalse{InString}% Assume not initally in string

\newcommand\digitstyle{\color{numeric}}
\makeatletter
\newcommand{\ProcessDigit}[1]
{%
  \ifnum\lst@mode=\lst@Pmode\relax%
   {\digitstyle #1}%
  \else
    #1%
  \fi
}
\makeatother

\lstset{literate=%
    {0}{{{\ProcessDigit{0}}}}1
    {1}{{{\ProcessDigit{1}}}}1
    {2}{{{\ProcessDigit{2}}}}1
    {3}{{{\ProcessDigit{3}}}}1
    {4}{{{\ProcessDigit{4}}}}1
    {5}{{{\ProcessDigit{5}}}}1
    {6}{{{\ProcessDigit{6}}}}1
    {7}{{{\ProcessDigit{7}}}}1
    {8}{{{\ProcessDigit{8}}}}1
    {9}{{{\ProcessDigit{9}}}}1
	{<=}{{\(\leq\)}}1
	{>=}{{\(\geq\)}}1,
	% morestring=[b]",
    % morestring=[b]',
    % morecomment=[l]//,
}

% Python style for highlighting
\newcommand\pythonstyle{\lstset{
language=Python,
basicstyle=\ttfamily\tiny,
numbers=left,
numberstyle=\tiny\color{linos},
morekeywords={self},              % Add keywords here
keywordstyle=\tiny\color{keywords},
commentstyle=\it\tiny\color{comments},    % Custom highlighting style
stringstyle=\tiny\color{strings},
xleftmargin=18pt,
xrightmargin=4pt,
aboveskip=0pt,
belowskip=0pt,
escapeinside={(*@}{@*)},
frame=l,                         % Any extra options here
showstringspaces=false,
keepspaces=true
}}

% Python environment 
\lstnewenvironment{python}[1][]
{
	\pythonstyle
	\lstset{
	#1
	}
}
{}

% wrap the Python environment
\newenvironment{codeblock}
    {\hfill\begin{beamerboxesrounded}[lower=codecol, width=0.8\textwidth]
    \medskip

    }
    { 
    \end{beamerboxesrounded}\hfill
    }

\theoremstyle{example}
\newtheorem{question}{Question}

\newcommand{\ct}[1]{\lstinline[language=Python]!#1!}
\newcommand{\ttt}[1]{{\small\texttt{#1}}}
\newcommand{\lsitem}[2]{\ttt{{#1}[}\ct{#2}\ttt{]}}

\author{Chris Cornwell}
\date{Jan 2, 2025}
\title{Comparisons and Control Flow}

\begin{document}

\begin{frame}
\titlepage
\end{frame}

\begin{frame}
\frametitle{Outline}
\tableofcontents
\end{frame}

\section{Comparisons}

%%%%
\begin{frame}
\frametitle{Comparisons}
We often need to compare variables, or compare a variable to some value {--} a ``literal.'' 

\pause
The comparison operations are in the table below.
\begin{center}
	\ttt{
	\begin{tabular}{|p{0.05\linewidth}|p{0.05\linewidth}|p{0.05\linewidth}|p{0.05\linewidth}|p{0.05\linewidth}|p{0.05\linewidth}|p{0.05\linewidth}|p{0.12\linewidth}|}
		\hline
		== & != & < & <= & > & >= & is & is not \\
		\hline
	\end{tabular}
	}
\end{center}
\begin{itemize}
	\item[] \ttt{==} and \ttt{!=}, for \emph{equals} and \emph{is not equal}.
	\item[] \ttt{<=}, for \emph{less than or equal to} (and similarly with \ttt{>=} and greater than).
\end{itemize}

\pause
\begin{itemize}
	\item Error occurs if a comparison doesn't make sense for that type. 
	\item Don't use \ttt{is} or \ttt{is not} with literals {--} meaning, they compare values of two variables but not, for example, if variable \ttt{x} equals \ttt{5}. Use either \ttt{==} or \ttt{!=} for that.
\end{itemize}

\end{frame}

%%%%
\begin{frame}[fragile]
\frametitle{Comparisons, examples}
Below, an example that returns True, but produces a syntax warning.
\vspace*{12pt}

\begin{codeblock}

\begin{python}
v, w = 10, 5
x = w+5
# A warning from next line; read warning message
print(x is 10)
\end{python}

\end{codeblock}

\vfill
\pause
With the same variable assignments as above, the following prints \ttt{True} twice.
\vspace*{12pt}

\begin{codeblock}

\begin{python}
print(v is x)
x == 10
\end{python}

\end{codeblock}
	
\end{frame}

\section{Control Flow}

%%%%
\begin{frame}
\frametitle{Control Statements}
So far, all code has had simple structure {--} a few lines of code each get executed once in order, top to bottom. Need a less simple structure to do interesting things. 
\begin{itemize}
    \item[] Achieve this with \textbf{control flow} statements.
\end{itemize}

\pause
Right now, we'll look at the following.
\begin{itemize}
	\item \ttt{if-else} statements.
    \pause
	\item Loops: \ttt{for} loops.
	\item Loops: \ttt{while} loops.
\end{itemize}
\end{frame}

\subsection{if-else statements}
%%%%
\begin{frame}[fragile]
\frametitle{{\ttm if-else} statements}

When you want part of code to execute only when a condition is satisfied, use an \ttt{if-else} statement. 

\pause
A basic example below (\ct{v=10} and \ct{w=5} as in slide from Comparisons).

\begin{codeblock}

\begin{python}
# basic if - else statement structure
Class = "MATH 371"
if len(Class) > v:
    y = w-5
    print("This sure is a long Class!")
else:
    y = w+10
    print("This Class just flies by!")
\end{python}

\end{codeblock}

\vspace*{12pt}

Notice the indentation of the lines between \ttt{if} and \ttt{else}, and those after \ttt{else}, which determines the code that runs on the condition. 
\pause
\begin{itemize}
	\item[] Lines 7 and 8 will only be executed if \ttt{len(Class)} is less than \ct{11}. If a line after line 8 is \emph{not} indented, it will be executed no matter what.
\end{itemize}

\end{frame}

%%%%
\begin{frame}[fragile]
\frametitle{{\ttm elif} and {\ttm pass}}

What if we have more than just two cases? That is, when the \ttt{if} condition fails we want to check another condition before deciding what to do.

\pause
\begin{itemize}
	\item[] Put an \ttt{elif} block between the \ttt{if} and \ttt{else}. (This is short for ``else, if\ldots''.)
\end{itemize}

\begin{codeblock}

\begin{python}
if y < w:
    print(f'y is less than w: {y} < {w}.')
elif y < 2*v:
    print(f'y >= w and less than 2v: {w} <= {y} < {2*v}.')
else:
    print(f'y >= 2v: {y} >= {2*v}.')
\end{python}

\end{codeblock}

\vfill
\pause
If we only care to do something in one of the cases, put \ttt{pass} in the block for the other case.

\begin{codeblock}

\begin{python}
if condition_to_be_checked:
    # some code here that does something
else:
    pass
\end{python}

\end{codeblock}
\end{frame}

\subsection{for loops}
%%%%
\begin{frame}[fragile]
\frametitle{Loops: {\ttm for} loop construction}

Use a \emph{loop} to construct, or compute, something through various iterations {--} the lines in a block of code execute over and over again until, at some point, it finishes.

The block of code that repeats should be indented, as with \ttt{if-else} statements. The next line not indented is outside that loop.

\pause
Example:

\begin{codeblock}

\begin{python}
# for loop that adds integers 1 through 5
the_sum = 0
for i in [1,2,3,4,5]:
    the_sum += i
print(the_sum)
\end{python}

\end{codeblock}

\pause
To avoid writing out the list of values for \ttt{i}, the following does the same as previous.

\begin{codeblock}

\begin{python}
the_sum = 0
for i in range(1,6):
    the_sum += i
print(the_sum)
\end{python}
	
\end{codeblock}
\begin{itemize}
    \item[] The next slide is about the \ttt{range()} function.
\end{itemize}
\end{frame}

%%%%
\begin{frame}[fragile]
\frametitle{More about {\ttm range()} constructor}
{\ttb range} is a type; it is similar to a list of ints. The function \ttt{range()} creates a range object (an instance of it). 

This function uses three arguments {--} \ttt{start}, \ttt{stop}, and \ttt{step}. %; however, \ttt{start} and \ttt{step} have a \emph{default value}, of \ct{0} and \ct{1}, if they are not provided. 
As a list, \ttt{range(start, stop, step)} will contain integers between \ttt{start} and \ttt{stop-1}, with a gap of \ttt{step} between consecutive items. 

\pause
For example, if \ttt{step = 1}, the range is 
\[ [\ttt{start}, \ttt{start+1},\ttt{start+2},\ldots,\ttt{stop-1}]. \]

\pause
More generally, its largest item equals \ttt{start + k*step}, with \ttt{k} as large as possible to remain less than \ttt{stop}. For example, \ttt{range(}\ct{2, 30, 4}\ttt{)} is \ct{[2,6,10,14,18,22,26]} as a list.

\pause
Given only two inputs, \ttt{range()} uses them as \ttt{start} and \ttt{stop}, setting \ttt{step=1}. A single input is interpreted as \ttt{stop}, setting \ttt{start=0}, \ttt{step=1}.\footnote{The behavior that has \ttt{start=0} and \ttt{step=1} here, we say these arguments have \emph{default} values of \ttt{0} and \ttt{1}.} So, \ttt{range(6)} is \ct{[0,1,2,3,4,5]}.
\end{frame}

%%%%
\begin{frame}[fragile]
\frametitle{Other sequence types and for loops}
\textbf{A great thing:} you are not \emph{required} to use some range object in a \ttt{for} loop. You can replace it with anything of sequential type. 
\begin{itemize}
	\item When it makes sense, can make code better for reading!
\end{itemize}

\vfill
\pause
Example: say that in the variable \ttt{Class}, assigned to be \ttt{"MATH 371"} before, we're inserting a period \ttt{"."} after each letter (but, not after a space). A for loop to do it is below.
\vspace*{12pt}

\begin{codeblock}

\begin{python}
new_Class_string = ""
for c in Class:
    if c == " ":
        new_Class_string += c
    else:
        new_Class_string += c + "."
print(new_Class_string)
\end{python}

\end{codeblock}
\end{frame}

\subsection{while loops}
%%%%
\begin{frame}[fragile]
\frametitle{Loops: {\ttm while} loop construction}

Rather than repeating a block of code for items in a specified sequence, we can simply repeat it until a condition is satisfied {--} a \ttt{while} loop.

\pause
\vspace*{12pt}
Above the code block to be repeated, put: \ttt{while <condition>}, where \ttt{<condition>} should evaluate to either \ttt{True} or \ttt{False}.

Example:
\vspace*{12pt}

\begin{codeblock}

\begin{python}
# Fibonacci numbers less than 10000
fibo_list = [1,1]
while fibo_list[-2] + fibo_list[-1] < 10000:
    fibo_list += [ fibo_list[-2] + fibo_list[-1] ]
print(fibo_list)
\end{python}

\end{codeblock}

\end{frame}

\section{List comprehensions}
%%%%
\begin{frame}[fragile]
\frametitle{Ways of constructing lists}

To generate or construct a list with a loop, a common way involves two parts.
\begin{enumerate}
	\item Make an empty list.
	\item Go through a loop, adding items to the list in each (or some) of the iterations.
\end{enumerate}

\pause
Example: have a function, \ttt{item()}, that determines the \ttt{n}$^{th}$ item in the list. You want list to contain the first 50 items. The code could look like below. 

\begin{codeblock}

\begin{python}
some_list = []
for n in range(50):
    # figure out the item with function item(n)
    some_list += [item(n)]
\end{python}

\end{codeblock}

\pause
In Python, the same list can be created in one line of code. You do so by typing the following.

\begin{codeblock}

\begin{python}[numbers=none]
some_list = [item(n) for n in range(50)]
\end{python}

\end{codeblock}

\end{frame}

%%%%
\begin{frame}[fragile]
\frametitle{Ways of constructing lists, continued}
In the previous slide, the variable \ttt{some}\ct{_}\ttt{list} is assigned by a \emph{list comprehension}.

\begin{codeblock}

\begin{python}[numbers=none]
some_list = [item(n) for n in range(50)]
\end{python}

\end{codeblock}

\pause
In list comprehensions: can add a condition to check. For example, if you only want to include \ttt{item(n)} if it is even, then you can write the following.

\begin{codeblock}

\begin{python}[numbers=none]
some_list = [item(n) for n in range(50) if item(n) % 2 == 0]
\end{python}

\end{codeblock}

\pause
\vfill
A more complicated example: get those $x$ between 1 and 100 at which the function $1000\frac{x}{(x^3+1)}$ has value between 0.5 and 2.

\begin{codeblock}

\begin{python}[numbers=none]
[x for x in range(1,101) if 0.5 < 1000*x/(x**3+1) < 2]
\end{python}

\end{codeblock}

\end{frame}

\end{document}