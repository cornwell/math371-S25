\documentclass{beamer}

\usepackage{helvet}
\usepackage{hyperref, graphicx}
\usepackage{amsthm}
\usepackage{etoolbox}
%\usepackage{multicol}
\usepackage{tikz}
\usepackage{ulem}

\usetheme{default}
\setbeamertemplate{navigation symbols}{}
\AtBeginSection[ ]
{
\begin{frame}{Outline}
    \tableofcontents[currentsection]
\end{frame}
}

% Default fixed font does not support bold face
\DeclareFixedFont{\ttb}{T1}{txtt}{bx}{n}{11} % for bold
\DeclareFixedFont{\ttm}{T1}{txtt}{m}{n}{12}  % for normal - use in headings

% Custom colors
\usepackage{color}
\definecolor{TUGray}{RGB}{101,101,137}
\definecolor{TUBlack}{RGB}{30,0,0}
\definecolor{mygreen}{RGB}{45,111,63}
\definecolor{keywords}{RGB}{205,114,0}
\definecolor{comments}{RGB}{181,51,139}
\definecolor{strings}{RGB}{58,144,81}
\definecolor{numeric}{RGB}{66,110,176}
\definecolor{linos}{rgb}{0.4,0.4,0.4}
\definecolor{links}{rgb}{0,0.4,0.75}

\definecolor{bggray}{RGB}{232, 233, 235}

\usecolortheme[named=mygreen]{structure}
\setbeamercolor{normal text}{fg=TUBlack}\usebeamercolor*{normal text}

\setbeamercolor{codecol}{fg=TUGray!25!black,bg=bggray}

\hypersetup{colorlinks, linkcolor=links, urlcolor=links}

\usepackage[T1]{fontenc}
\usepackage[sfdefault,scaled=.85]{FiraSans}
\usepackage{newtxsf}

\usepackage{listings}

\newtoggle{InString}{}% Keep track of if we are within a string
\togglefalse{InString}% Assume not initally in string

\newcommand\digitstyle{\color{numeric}}
\makeatletter
\newcommand{\ProcessDigit}[1]
{%
  \ifnum\lst@mode=\lst@Pmode\relax%
   {\digitstyle #1}%
  \else
    #1%
  \fi
}
\makeatother

\lstset{literate=%
    {0}{{{\ProcessDigit{0}}}}1
    {1}{{{\ProcessDigit{1}}}}1
    {2}{{{\ProcessDigit{2}}}}1
    {3}{{{\ProcessDigit{3}}}}1
    {4}{{{\ProcessDigit{4}}}}1
    {5}{{{\ProcessDigit{5}}}}1
    {6}{{{\ProcessDigit{6}}}}1
    {7}{{{\ProcessDigit{7}}}}1
    {8}{{{\ProcessDigit{8}}}}1
    {9}{{{\ProcessDigit{9}}}}1
	{<=}{{\(\leq\)}}1
	{>=}{{\(\geq\)}}1,
	% morestring=[b]",
    % morestring=[b]',
    % morecomment=[l]{//},
}

\lstdefinelanguage{Pseudo}{
    morekeywords={return, while, if, for, input},
    morecomment=[l]{\#},
}

% Pseudocode style
\newcommand\pseudostyle{\lstset{
language=Pseudo,
basicstyle=\fontfamily{ccr}\scriptsize,
commentstyle=\it\scriptsize\color{linos},
keywordstyle=\it\bfseries\scriptsize,
mathescape=true,
literate=
    {=}{$\leftarrow{}$}{1}
    {==}{$={}$}{1}
    {<=}{{\(\leq\)}}1
	{>=}{{\(\geq\)}}1,
xleftmargin=18pt,
xrightmargin=4pt,
aboveskip=12pt,
belowskip=0pt,
frame=tB,
keepspaces=true
}}

% Python style for highlighting
\newcommand\pythonstyle{\lstset{
language=Python,
basicstyle=\ttfamily\tiny,
numbers=left,
numberstyle=\tiny\color{linos},
morekeywords={self, np},              % Add keywords here
keywordstyle=\tiny\color{keywords},
commentstyle=\it\tiny\color{comments},    % Custom highlighting style
stringstyle=\tiny\color{strings},
xleftmargin=18pt,
xrightmargin=4pt,
aboveskip=0pt,
belowskip=0pt,
escapeinside={(*@}{@*)},
frame=l,                         % Any extra options here
showstringspaces=false,
keepspaces=true
}}

% Pseudocode environment
\lstnewenvironment{pseudo}[1][]
{
    \pseudostyle
    \lstset{
        #1
    }
}
{}

% Python environment 
\lstnewenvironment{python}[1][]
{
	\pythonstyle
	\lstset{
	#1
	}
}
{}

% wrap the Python environment
\newenvironment{codeblock}
    {\hfill\begin{beamerboxesrounded}[lower=codecol, width=0.8\textwidth]
    \medskip

    }
    { 
    \end{beamerboxesrounded}\hfill
    }

\theoremstyle{example}
\newtheorem{question}{Question}

\newcommand{\ct}[1]{\lstinline[language=Python]!#1!}
\newcommand{\ttt}[1]{{\small\texttt{#1}}}
\newcommand{\lsitem}[2]{\ttt{{#1}[}\ct{#2}\ttt{]}}

\author{Chris Cornwell}
\date{Mar 13, 2025}
\title{Overview of Machine Learning \newline 
    \footnotesize{in particular, Supervised Learning}}

\begin{document}

\begin{frame}
\titlepage
\end{frame}

\begin{frame}
\frametitle{Outline}
\tableofcontents
\end{frame}

\section{Machine Learning}

%%%%
\begin{frame}
\frametitle{What is Machine Learning?}
    % Mitchell '98 ``Well posed learning problem''
    Definition by Tom Mitchell:\newline 
    \begin{quote}
        A ``computer program'' is said to \textbf{learn} from experience $E$, with respect to some task $T$ and performance measure $P$ if: its performance on $T$, as measured by $P$, improves with experience $E$.
    \end{quote}
    \pause
    \begin{itemize}
        \item The definition is intentionally general. Often, could think of $E$ as ``training'' (updates to how program runs), based on observed data.
        \pause
        \item ``computer program,'' for us, means a function implemented on a computer that produces output from given input. The output is how the program achieves the task $T$. 
        \pause
        \item The procedures discussed in class {--} linear regression and the Perceptron algorithm for half-space model {--} fit into this paradigm\ldots \textit{kind of}.
    \end{itemize}

\end{frame}

%%%%
\begin{frame}
\frametitle{What is Machine Learning?}
    Definition by Tom Mitchell:\newline 
    \begin{quote}
        A computer program is said to \textbf{learn} from experience $E$, with respect to some task $T$ and performance measure $P$ if: its performance on $T$, as measured by $P$, improves with experience $E$.
    \end{quote}
    
    {\color{mygreen}Examples:}
    \begin{enumerate}
        \item Linear regression.
        \begin{itemize}
            \item Output of $\hat{y}$ on input $x$ (potentially multiple variables).
            \pause
            \item $T$: fit observed points $\{(x_i,y_i)\}_{i=1}^n$ well with predictions $\{(x_i,\hat{y}_i)\}$, where $\hat{y}_i=mx_i+b$ for some $m,b$ (an expectation of $(x,\hat{y})$ being good fit on \textit{unobserved} data. 
            \pause
            \item $E$: ?? \newline 
                    The data are used to get $m$ and $b$, but you don't really ``improve'' with repeated use of data. \newline 
                    \pause
                    \emph{Closed form} for best choice of $m, b$, computing $(A^TA)^{-1}A^T{\bf y}$.
            \pause
            \item $P$: Mean squared error.
        \end{itemize}
        Having closed form, result of simplicity of the form of $\hat{y}_i$.
    \end{enumerate}

\end{frame}

%%%%
\begin{frame}
\frametitle{What is Machine Learning?}
    Definition by Tom Mitchell:\newline 
    \begin{quote}
        A computer program is said to \textbf{learn} from experience $E$, with respect to some task $T$ and performance measure $P$ if: its performance on $T$, as measured by $P$, improves with experience $E$.
    \end{quote}
    
    {\color{mygreen}Examples:}
    \begin{enumerate}
        \setcounter{enumi}{1}
        \item The Perceptron algorithm.
        \begin{itemize}
            \item Output of label $\pm 1$ on input ${\bf x}\in \mathbb R^d$ (or something \textit{turned into} ${\bf x}\in\mathbb R^d$).
            \pause
            \item $T$: predict labels correctly, using $W = ({\bf w}, b)\in\mathbb R^{d+1}$ to decide label, $y = \textrm{sign}({\bf w}\cdot{\bf x} + b)$ \newline 
            \ldots hopefully works on \textit{unobserved} data.
            \pause
            \item $E$: looking through observed data $X_i = ({\bf x}_i,1)$, label $y_i$, and updating $W^{(t+1)} = W^{(t)} + y_iX_i$ when $i$ found with $W^{(t)}\cdot(y_iX_i)\le 0$.
            \pause
            \item $P$: ?? \newline 
                    Whether its labels on all observed data are correct. But, only two results: \textit{True} or \textit{False}. \newline 
                    \pause
                    If data is linearly separable, enough of experience $E$ improves this measure (changing to \textit{True}). Only happens if linearly separable.
        \end{itemize}
    \end{enumerate}

\end{frame}

%%%%
\begin{frame}
\frametitle{What are the general types of \emph{tasks} in machine learning?}
Supervised learning: the program learns from sample data that has labels. Goal: determine underlying function from sample data.\newline
\pause
{\color{mygreen}Examples.}
    \begin{itemize}
        \item Housing price prediction
        \item Whether emails are phishing or not phishing.
        \item Determine if a satellite image of ocean has floating trash.
        \item Try to auto-complete a sentence being typed.
    \end{itemize}
\pause
Unsupervised learning: there is sample data, but the data does not have any labels. Goal: discover something (a pattern, grouping, or some insight) about the data.\newline
\pause
{\color{mygreen}Examples.}
    \begin{itemize}
        \item Market segmentation.
        \item News feed (grouping similar news articles).
        \item Separate audio sources in a mixed signal.
        \item Organize computing clusters.
    \end{itemize}

\end{frame}

\section{Supervised learning}

%%%%
\begin{frame}
\frametitle{The goal of supervised learning}
Have an ``input space'' (which often is $\mathbb R^{d}$, or a subset of it, but could be a different space); and have an output space, or label space, $Y$. 

\pause
\begin{itemize}
    \item Given a sample $\mathcal S = \{({\bf x}_i, y_i)\}_{i=1}^n$, with ${\bf x}_i\in\mathbb R^d$ and $y_i\in Y$, drawn from an (unknown) joint probability distribution $P_{X,Y}:\mathbb R^{d}\times Y \to [0, \infty)$. 
    \pause
    \item Goal: to learn, from $\mathcal S$, a function $f^*:\mathbb R^d\to Y$ that ``fits'' (\textit{approximates well}) the distribution $P_{X,Y}$. 
    \pause
    \item You might not be able to have points on the graph of $f^*$ be typically ``very close'' to samples from $P_{X,Y}$. However, ideally, for an ${\bf x}\in\mathbb R^d$ corresponding $y$-value on graph is near the expected value given ${\bf x}$.
\end{itemize}

\end{frame}

%%%%
\begin{frame}
\frametitle{How to achieve the goal}
Most often, we choose a \textit{parameterized class} of functions\footnote{Sometimes called a \textit{hypothesis class}.}, and we get $f^*$ from that class. 
\pause
\begin{itemize}
    \item That is, there is a space of parameters $\Omega$; an $\omega\in\Omega$ determines a function $f_{\omega}:\mathbb R^d \to Y$, and the parameterized class is the set of all such functions $f_\omega$. % Model selection
    \pause
    \item To learn a function that fits well, you look for good parameters.
\end{itemize}

\pause
How do we find good parameters?

Select a performance measure: \textbf{(empirical) loss function} $\mathcal L_{\mathcal S}:\Omega \to \mathbb R$. In the empirical loss function, we use $\mathcal S$ in its definition.
\pause
\begin{itemize}
    \pause
    \item Then, $\mathcal L_{\mathcal S}$ is used to determine how to make changes to parameters, $\omega$, in order to decrease the value of $\mathcal L_{\mathcal S}$.
    \pause
    \item In an ideal situation, you converge to some $\omega^*$, a minimizer of $\mathcal L_{\mathcal S}$, and set $f^* = f_{\omega^*}$.
\end{itemize}

\end{frame}

%%%%
\begin{frame}
    \frametitle{For linear regression}
    Have sample data $\mathcal S$, with data points $x_i$ in $\mathbb R$ (so, $d=1$). The parameter space $\Omega = \mathbb R^2 = \{(m,b)\ |\ m\in\mathbb R, b\in\mathbb R\}$. For each $\omega = (m,b)$, we have
        \[f_{\omega}(x) = mx + b.\]
    \pause
    Loss function: the MSE. That is, set 
        \[\mathcal L_{\mathcal S}(m,b) = \frac1{n}\sum_{i=1}^n (mx_i + b - y_i)^2.\]
\end{frame}

\section{First look at Gradient Descent}

%%%%
\begin{frame}
\frametitle{Gradient descent with simple linear regression}
For $\omega = (m, b)$, have $f_{\omega} = mx + b$.
\pause 

\end{frame}

\end{document}